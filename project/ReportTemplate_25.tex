% !TeX spellcheck = en_GB
% Template for Federated Learning Course Project Report

\documentclass[9pt]{article}
\usepackage{spconf,amsmath,bm,graphicx}
\usepackage[a4paper, margin=1in]{geometry}
\usepackage[dvipsnames]{xcolor}
\usepackage{hyperref, cleveref, tcolorbox}
\usepackage{algorithm, algorithmic}
\usepackage{amsfonts,amssymb,amsbsy}
\usepackage{subcaption, adjustbox}

% Declare Math Operators
\DeclareMathOperator*{\argmax}{arg\,max}
\DeclareMathOperator*{\argmin}{arg\,min}

% Title
\title{Federated Learning Project Report Template}
\name{First Last, \texttt{email@address.com}}
\address{}

\begin{document}
	\maketitle
	
	\begin{abstract}
		This abstract provides a concise summary of the project, including the FL application, empirical graph modeling, variation minimization approach, and the FL algorithms used.
	\end{abstract}
	
	\textbf{Keywords:} Federated Learning, Networks, Personalized ML, Trustworthy AI
	
	\section{Introduction}
	\label{sec:intro}
	{\bf Instructions (Remove before submission)]
		Introduce the background and motivation for your FL project:
		\begin{itemize}
			\item A real-life scenario motivating your FL application.
			\item Summary of state-of-the-art methods relevant to your project.
			\item Brief outline of the structure of your report.
		\end{itemize}}
	
	\section{Problem Formulation}
	\label{sec:pf}
	{\bf Instructions (Remove before submission)]
		Model your FL application as a FL network (see \cite[Ch.~3]{Jung2025}). 
		In particular, clearly define and explain:
		\begin{itemize}
			\item Nodes: What real-world devices do they represent?
			\item Local Models: Describe the models used at each node.
			\item Loss Functions: Specify local loss functions for training.
			\item Edges: How are edges and their weights chosen?
		\end{itemize}}
	
	\section{Methods}
	\label{sec:methods}
{\bf Instructions (Remove before submission)]
	The project requires you to apply GTVMin-based methods to 
	the FL application modelled in Section \ref{sec:pf}. In this section 
	you need to clearly state and explain:
		\begin{itemize}
			\item Your choice of variation measure.
			\item Your choice of FL algorithm and its message passing implementation.
		\end{itemize}}
	
	\section{Numerical Experiments}
	\label{sec:experiments}
	{\bf Instructions (Remove before submission)]
		Describe implementation details and analyze results:
		\begin{itemize}
			\item Data sources used.
			\item Model validation, selection, and diagnosis methods (see \cite[Sec.~6.6]{Jung2022}).
			\item Report training, validation, and test losses.
		\end{itemize}}
	

\textbf{Important:} Your report must include a zip archive containing a single 
Python script along with any necessary data files. Minimize the use of non-standard 
Python packages to ensure ease of execution and reproducibility.

	
	\section{Conclusion}
	\label{sec:conclusion}
	{\bf Instructions (Remove before submission)
		\begin{itemize}
			\item Discuss whether the obtained results solve the problem satisfactorily.
			\item Identify limitations and suggest potential improvements.
		\end{itemize}}
	
%	\section*{References}
	\begin{thebibliography}{9}
		\bibitem{Jung2025} A. Jung, \textit{Federated Learning: From Theory to Practice}, Aalto, 2025. Available: \url{https://github.com/alexjungaalto/FederatedLearning/blob/main/material/FLBook.pdf}.
		\bibitem{Jung2022} A. Jung, \textit{Machine Learning: The Basics}, Springer, 2022.
	\end{thebibliography}
	
\end{document}